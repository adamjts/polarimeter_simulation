\documentclass[11pt, oneside]{ltxdoc}   	% use "amsart" instead of "article" for AMSLaTeX format
\usepackage{geometry}                		% See geometry.pdf to learn the layout options. There are lots.
\geometry{letterpaper}                   		% ... or a4paper or a5paper or ... 
%\geometry{landscape}                		% Activate for rotated page geometry
%\usepackage[parfill]{parskip}    		% Activate to begin paragraphs with an empty line rather than an indent
\usepackage{graphicx}				% Use pdf, png, jpg, or eps§ with pdflatex; use eps in DVI mode
								% TeX will automatically convert eps --> pdf in pdflatex		
\usepackage{amssymb}

%SetFonts

%SetFonts


\title{ComponentGroups.py Library}
\author{Adam Theriault-Shay}
\date{Summer UROP 2016}							% Activate to display a given date or no date

\begin{document}
\maketitle
\section{Overview}
The ComponentGroups.py library is designed to make running polarimetry simulations quick and simple. Long and high resolution simulations can be run with just four commands at the command line. Right now the library supports simulations that run from a point source to a multi-layer mirror, then to another multi-layer mirror, and finally to a CCD detector. The dimensions and spacing of all components can be customized via functions included in the library.\\
When running simulations, the library will also generate a file structure in which trial data is stored. The library also includes objects that generate relevant plots and histograms.\\
It has the following dependencies: matplotlib, numpy, astropy, transforms3d, marxs.

\section{Simulations}
\subsection{Overview}
\section{Data and Graphing}



\end{document}  